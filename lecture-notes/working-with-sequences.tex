\subsection{Working with sequences}\label{subsec:sequences}
\elenacomment{Might want to break this into subsubsections and name better; perhaps make into a separate section}

Clojure has numerous functions that work with sequences in a \fut{collection-independent} way: they may take in a sequence or sequences and may return a sequence. We have seen \clocode{add-first}, \clocode{add-last}, and \clocode{concat} that follow this setup. In this section we will examine a few more of such functions (focusing in their collection-independent nature), and then introduce more \elenacomment{perhaps not in this subsection}. These functions are widely used and will be very helpful for any work you do in Clojure. 

We start by introducing a function \clocode{count}. It takes in any collection and returns the number of elements in it:
\begin{framed}
\begin{verbatim}
intro.core=> (count '(5 2 7))
3
intro.core=> (count [1 4 1 7])
4
intro.core=> (count [])
0
intro.core=> (count (rest [1 4 1 7]))
3
intro.core=> (count [[1 2] "Hello" false])
3
\end{verbatim}
\end{framed}
Note that the function does not know anything about what the elements are. For example, if an element is itself a vector or a list, it is counted as one element. 
%\elenacomment{Perhaps this needs to be explained when we talk about "first"}

There are many functions that take in a sequence (or several) and return a sequence. As an example, consider \clocode{take}. The function takes a number $n$ and a collection and returns the first $n$ items in the collection, or all the items in the collection if it has $n$ or fewer elements. Let's look at some examples:
\begin{framed}
\begin{verbatim}
intro.core=> (take 5 '(10 9 8 7 6 5 4 3 2 1))
(10 9 8 7 6)
intro.core=> (take 3 ["a" "b" "c" "d" "e"])
("a" "b" "c")
intro.core=> (take 5 '(3 2 1))
(3 2 1)
intro.core=> (take 3 [])
()
intro.core=> (take 3 '())
()
\end{verbatim}
\end{framed} 
The interesting thing to observe is that, just as \clocode{rest} and \clocode{concat}, the function \clocode{take} always returns a sequence. 

Now that we have seen this pattern of taking in a collection (or collections) and returning a sequence many times, let us examine it closer. Recall that the term \fut{collection} refers to a specific way of storing items, a particular type of a \fut{"container"}. A \fut{sequence} is an abstraction that refers to accessing elements in a specific order. If I have a vector \clocode{[3 2 5]} stored in memory, its items, when accessed as a sequence, should appear in the order $3, 2, 5$. In other words, the \fut{sequence} \clocode{(3 2 5)} corresponds to the vector \clocode{[3 2 5]}. If I call \clocode{first} on a vector  \clocode{[3 2 5]}, it produces a sequence corresponding to the vector and then takes its first element. 

This procedure seems almost trivial for vectors and lists, but it's important to realize that this is how collection-independent functions work on any collection: they first take the sequence that corresponds to the collection, and then apply the required function to it. This means that \clocode{count} first produces a sequence corresponding to the collection passed to it, and then counts the elements in that sequence. Likewise, \clocode{concat} concatenates sequences corresponding to the two given collections into one. 

It is reasonable to expect that Clojure has a function that produces a sequence corresponding to a collection. Such a function indeed exists and is called \clocode{seq}. It works as follows:
\begin{framed}
\begin{verbatim}
intro.core=> (seq '(3 4 2))
(3 4 2)
intro.core=> (seq [3 4 2])
(3 4 2)
intro.core=> (seq '([1 2] 7 true))
([1 2] 7 true)
\end{verbatim}
\end{framed} 

\elenacomment{nil acts as an empty sequence; also in functions such as concat}
\begin{framed}
\begin{verbatim}
intro.core=> (seq [])
nil
intro.core=> (seq '())
nil
\end{verbatim}
\end{framed} 

Sequences generally act like lists. 

\elenacomment{should have links to documentation and the like}

\elenacomment{Mention how to create a list or a vector of a bunch of elements or a sequence}

\elenacomment{introduce \clocode{seq?}}

\subsection{In-depth: implementation of lists and vectors}\label{subsec:implement}


\subsection{Section resources:}
\begin{itemize}
\item \url{http://bpeirce.me/clojure-sequence-implementations.html}
\end{itemize}

