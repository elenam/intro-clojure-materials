\subsection{Working with sequences}\label{subsec:sequences}
\elenacomment{Might want to break this into subsubsections and name better; perhaps make into a separate section}
Clojure has numerous functions that work with sequences in a \fut{collection-independent} way: they may take in a sequence or sequences and may return a sequence. We have seen \clocode{add-first}, \clocode{add-last}, and \clocode{concat} that follow this setup. In this section we will examine a few more of such functions, focusing in their collection-independent nature. These functions are widely used and will be very helpful for any work you do in Clojure. 

It is useful to keep in mind that for all practical purposes sequences act and look like lists. Regardless of the way they are stored, sequences can be viewed as devices that produce elements in a specific order. Note that sequences don't have to be finite: we will see \fut{lazy} sequences in Clojure that have no limit on the number of elements they produce \elenacomment{Reference}. However, at this point we will only work with finite sequences. 

We start by introducing a function \clocode{count}. It takes in any collection and returns the number of elements in it:
\begin{framed}
\begin{verbatim}
intro.core=> (count '(5 2 7))
3
intro.core=> (count [1 4 1 7])
4
intro.core=> (count [])
0
intro.core=> (count (rest [1 4 1 7]))
3
intro.core=> (count [[1 2] "Hello" false])
3
\end{verbatim}
\end{framed}
Note that the function does not know anything about what the elements are. For example, if an element is itself a vector or a list, it is counted as one element. 
%\elenacomment{Perhaps this needs to be explained when we talk about "first"}

There are many functions that take in a sequence (or several) and return a sequence. As an example, consider \clocode{take}. The function takes a number $n$ and a collection and returns the first $n$ items in the collection, or all the items in the collection if it has $n$ or fewer elements. Let's look at some examples:
\begin{framed}
\begin{verbatim}
intro.core=> (take 5 '(10 9 8 7 6 5 4 3 2 1))
(10 9 8 7 6)
intro.core=> (take 3 ["a" "b" "c" "d" "e"])
("a" "b" "c")
intro.core=> (take 5 '(3 2 1))
(3 2 1)
intro.core=> (take 3 [])
()
intro.core=> (take 3 '())
()
\end{verbatim}
\end{framed} 
The interesting thing to observe is that, just as \clocode{rest} and \clocode{concat}, the function \clocode{take} always returns a sequence. 

Now that we have seen this pattern of taking in a collection (or collections) and returning a sequence many times, let us examine it closer. Recall that the term \fut{collection} refers to a specific way of storing items, a particular type of a \fut{"container"}. A \fut{sequence} is an abstraction that refers to accessing elements in a specific order. If I have a vector \clocode{[3 2 5]} stored in memory, its items, when accessed as a sequence, should appear in the order $3, 2, 5$. In other words, the \fut{sequence} \clocode{(3 2 5)} corresponds to the vector \clocode{[3 2 5]}. If I call \clocode{first} on a vector  \clocode{[3 2 5]}, it produces a sequence corresponding to the vector and then takes its first element. 

This procedure seems almost trivial for vectors and lists, but it's important to realize that this is how collection-independent functions work on any collection: they first take the sequence that corresponds to the collection, and then apply the required function to it. This means that \clocode{count} first produces a sequence corresponding to the collection passed to it, and then counts the elements in that sequence. Likewise, \clocode{concat} concatenates sequences corresponding to the two given collections into one. 

It is reasonable to expect that Clojure has a function that produces a sequence corresponding to a collection. Such a function indeed exists and is called \clocode{seq}. It works as follows:
\begin{framed}
\begin{verbatim}
intro.core=> (seq '(3 4 2))
(3 4 2)
intro.core=> (seq [3 4 2])
(3 4 2)
intro.core=> (seq '([1 2] 7 true))
([1 2] 7 true)
\end{verbatim}
\end{framed} 
Since sequences look and act like lists, the result of applying \clocode{seq} to a list doesn't change anything (at least as far as we can observe at this point). Applying \clocode{seq} to a vector returns a sequence of all its elements in order. The last example illustrates that \clocode{seq} returns all the elements of the original collection without descending into its sub-collections, such as the nested vector \clocode{[1 2]} which is returned as is. 

Now that we have introduced \clocode{seq}, we can observe that \clocode{take} on a vector can be viewed as following: first it makes a sequence out of a vector by applying \clocode{seq} to the vector, and then it takes the first $n$ elements of the resulting sequence. The following function calls illustrates how the call \clocode{(take 3  ["a" "b" "c" "d" "e"])} works behind the scenes:
\begin{framed}
\begin{verbatim}
intro.core=> (take 3  (seq ["a" "b" "c" "d" "e"]))
("a" "b" "c")
\end{verbatim}
\end{framed} 
Clojure performs the call to \clocode{seq} automatically, as a part of \clocode{take}, so you never need to make this call explicitly. 

In a similar fashion, a call \clocode{ (concat [1 2 3] '(4 5))} is equivalent to \clocode{(concat (seq [1 2 3]) (seq '(4 5)))}, so \clocode{concat} applies \clocode{seq} to both (or all, if more than two) its arguments, and then concatenates the resulting sequences. Examining this further, we realize that \clocode{first} and \clocode{rest} also turn their argument into a sequence, and then get the result. That's why \clocode{rest} always returns a sequence. Of course, at this point we can be guessing about this behavior since vectors and lists are so simple that we don't have any evidence that they are converted to sequences before these functions are applied. Our study of other types of collections will confirm this guess. 

Also note that \clocode{add-first} and \clocode{add-last} first produce a sequence out of their first argument, and then add the element to the resulting sequence at the appropriate position. 

The \clocode{seq} function returns a sequence that looks like a list, with one important exception: it returns \clocode{nil} as an empty sequence.
\begin{framed}
\begin{verbatim}
intro.core=> (seq [])
nil
intro.core=> (seq '())
nil
\end{verbatim}
\end{framed} 
This doesn't create any problems because \clocode{nil} acts like an empty sequence in all functions that take sequences:
\begin{framed}
\begin{verbatim}
intro.core=> (count nil)
0
intro.core=> (concat '(7 8) nil)
(7 8)
intro.core=> (concat nil nil)
()
intro.core=> (take 2 nil)
()
\end{verbatim}
\end{framed} 
The reason an empty sequence is represented by \clocode{nil} and not \clocode{'()} is because it makes checking for an empty sequence easier in some Clojure programs. We are not going into these details right now. 

\subsubsection{More on relation between sequences and collections}\label{subsub:seq-coll-convert}
Examining even further what is a collection and what is a sequence, we use two Clojure predicates \elenacomment{make sure to define the term earlier; need an index}: \clocode{coll?} that returns \clocode{true} if its argument is a collection and \clocode{false} otherwise, and \clocode{seq?} that returns \clocode{true} if its argument is a sequence and \clocode{false} otherwise. 

As always, we do a bit of exploring in REPL:
\begin{framed}
\begin{verbatim}
intro.core=> (coll? [1 true "apple"])
true
intro.core=> (coll? '(1 true "apple"))
true
intro.core=> (coll? [])
true
intro.core=> (coll? '())
true
intro.core=> (seq? [1 true "apple"])
false
intro.core=> (seq? '(1 true "apple"))
true
intro.core=> (seq? [])
false
intro.core=> (seq? '())
true
\end{verbatim}
\end{framed} 
We discover that lists and vectors (empty and non-empty) are collections. Lists are also sequence, but vectors aren't. However, the result of applying \clocode{seq} to a non-empty vector is a sequence:
\begin{framed}
\begin{verbatim}
intro.core=> (seq? (seq [1 true "apple"]))
true
\end{verbatim}
\end{framed} 
What about the result of applying \clocode{seq} to empty vectors or lists? The result of such application is \clocode{nil}, and, as we see below, \clocode{nil} is not a sequence:
\begin{framed}
\begin{verbatim}
intro.core=> (seq? nil)
false
intro.core=> (seq? (seq []))
false
intro.core=> (seq? (seq '()))
false
\end{verbatim}
\end{framed} 
The last two tests are actually unnecessary because we have already observed that \clocode{seq} of an empty collection is \clocode{nil} and we have just established that \clocode{nil} is not a sequence. Note, however, that \clocode{(empty? nil)} returns \clocode{true}, and so does \clocode{empty?} applied to any empty collection or sequence, and we will be using \clocode{empty?} in our programs to check for an empty sequence.

Of course, Clojure has plenty of things that aren't collections or sequences, such as numbers, booleans, and functions:
\begin{framed}
\begin{verbatim}
intro.core=> (coll? 5)
false
intro.core=> (seq? 5)
false
intro.core=> (coll? true)
false
intro.core=> (seq? false)
false
intro.core=> (coll? +)
false
intro.core=> (seq? +)
false
\end{verbatim}
\end{framed} 
Since numbers, booleans, and functions aren't collections, you cannot apply \clocode{seq} to them either. For instance, \clocode{(seq 5)} gives an error \cloerr{Don't know how to create a sequence from a number}. This error message now should be very clear: \clocode{5} is not already a sequence, so Clojure is trying to create one, but sequences can only be created from collections, and \clocode{5} is a number, not a collection. Note that you are going to get a similar error message if you are trying to pass a number (or a boolean or a function) to a function that takes a sequence or a collection. For instance, the same error  \cloerr{Don't know how to create a sequence from a number} happens when one of the arguments of \clocode{concat} is not sequence or a collection: \clocode{(concat [1 2] 5)}. This confirms our guess that \clocode{concat} creates a sequence out of its arguments (if they aren't sequences already) before concatenation. 

You may have noticed that we haven't shown any examples of strings in relation to sequences and collections. This is because strings are in a class of their own in this regard. We discuss strings and characters in relation to collections and sequences in section~\ref{subsec:str}.


Just like there is a way to create 

%\elenacomment{introduce \clocode{seq?} and \clocode{coll?}}

\elenacomment{Mention how to create a list or a vector of a bunch of elements or a sequence}

\elenacomment{May need to fix a ref on how to convert a sequence to a vector, to point to this sec}

%\elenacomment{nil acts as an empty sequence; also in functions such as concat}

\subsection{Working with sequence functions}\label{subsec:seq-functions}

\elenacomment{program examples}

\elenacomment{should have links to documentation and the like}


\subsection{In-depth: implementation of lists and vectors}\label{subsec:implement}

\subsection{Strings in relation to collections and sequences}\label{subsec:str}
\elenacomment{A string is not a collection, but it's seqable - not sure where that should be covered. Probably introduce strings earlier, but relate them to vectors and lists after the section on vectors and lists in a separate section}

\elenacomment{We may need to change the section title}

\subsection{Section resources:}
\begin{itemize}
\item \url{http://bpeirce.me/clojure-sequence-implementations.html}
\end{itemize}

