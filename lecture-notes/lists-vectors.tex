\section{Introduction to Lists and Vectors}\label{sec:list-vectors}
\subsection{Introduction to Clojure collections and sequences}\label{subsec:coll-seq}
Most real-life programs work with data: they analyze data, search it for elements with specific properties, transform it from one form to another for storage or display, and perform other operations. Data may be coming in from a database, from a user input or a web form, from traversing the web,  from another program, to name just a few sources. Every data item may be as simple as one number, and as complicated as contents of a book. While data is being processed by a program, it needs to be stored in the program's memory in some way that allows accessing elements, in some sort of virtual \fut{container}. 

Clojure  \fut{collections} are ``containers'' for storing data. There many types of collections, each suited to store data for a specific set of purposes. Every type of collections has a particular implementation in computer memory: its elements are stored in a specific way. Some collections allow faster access to data, some allow better ways of indexing it, some store data in a specific order, and some in a set-like fashion in which the order of elements does not matter. 
Since Clojure is a functional language, Clojure collections are \fut{immutable}. This means that any function that ``adds'' elements to a collection or ``removes'' elements from it return a \fut{new} collection instead of modifying the original one.  

If you read Clojure documentation or examples, you will notice that people frequently mention \fut{sequences}. Sometimes the terms ``collection'' and ``sequence"' seem to be used interchangeably. While there are many situations in which you can refer to the same data container as a collection or a sequence, they are not the same. A \fut{sequence} is not any concrete collection, but an abstraction. It refers to the ability to get elements from a collection (as well as from other sources, such as user input) in a \fut{sequential} manner: a fancy word for getting elements \fut{one at a time}. It does not matter how elements are stored internally: if you get the same elements in the same order from two containers, they are equal as sequences.

We will discuss collections and sequences more in the future, but for now remember that while both collections and sequences are used to describe data containers, \takeaway{collections refer to a concrete way of storing data in memory} and \takeaway{sequences are an abstraction for accessing elements in a sequential manner}. 

The two types of collections that we introduce at this point are \fut{lists} and \fut{vectors}. We explain some of the differences between the two and reasons for these differences. 

Both lists and vectors allow one to store data (of any type) in some order. Historically, lists have been the main data storage in the Lisp family of programming languages. They have a straightforward internal implementation. Vectors are unique to Clojure and are designed to be more efficient. Some details of the implementation of lists and vectors will be discussed later \elenacomment{Reference + more reading?}. 

Lists store their items in way that acts like a stack on plates: new items are added at ``the top'', and when one needs to take an item, they take it from ``the top''.  Of course, there is really no top in the machine representation of memory, but it is a convenient analogy.  This way of accessing elements in called \fut{last-in-first-out} and is abbreviated as \fut{LIFO}. Lists historically have been the main data structure in the Lisp family of languages. 

Vectors store their items similarly to a queue in a cafeteria: items are added at ``the end'' and removed from ''the beginning''. Once again, ``the end'' and removed from ''the beginning'' are just a convenient analogy, and do not correspond to the internal implementation. This way of accessing elements in often called \fut{first-in-first-out} and is abbreviated as \fut{FIFO}. 

\subsection{The very basics of lists}\label{subsec:lists}
Examples below show how to create a list and to work with it. Note that the single quotation mark in front of the list is needed: without it the parentheses would be interpreted as a function application. However, when the interpreter evaluates the list, the quotation mark is not printed. One can define an empty list, i.e. a list with no elements at all, by not putting any elements in the list: \clocode{'()}.
\begin{framed}
\begin{verbatim}
intro.core=> '(1 2 3)  ; creating a list
(1 2 3)
intro.core=> '(1 true 8.5) ; we can have anything in a list
(1 true 8.5)
intro.core=> (def mylist '(1 2 3)) ;saving a list in a variable
#'intro.core/mylist
intro.core=> mylist
(1 2 3)
intro.core=> (def emptylist '()) ; create an empty list
#'intro.core/emptylist
intro.core=> emptylist
()
\end{verbatim}
\end{framed}
A subtle point is that in Clojure you don't need to put the quotation mark in front of an empty list: \clocode{(def emptylist ())} works just fine. I will, for the most part, stick to using the quotation mark for uniformity, but feel free to skip it in your code if you prefer. Note, however, that forgetting the quotation mark for a non-empty list is an error: \clocode{ (def trylist (1 2 3))} results in the error \cloerr{Attempted to use a number, but a function was expected}. This is because \clocode{(1 2 3)} is interpreted as \clocode{1} applied to the arguments \clocode{2 3}, but \clocode{1} is a number, not a function. 

There are many functions that operate on lists (most of them can also operate on other collections). The most fundamental ones are \clocode{first} that returns the first element in a list and \clocode{rest} that returns the list of all elements of the list except the first one. The rest of a one-element list is an empty list. Continuing our example: 
\begin{framed}
\begin{verbatim}
intro.core=> (first mylist)
1
intro.core=> (rest mylist)
(2 3)
intro.core=> mylist
(1 2 3)
intro.core=> (rest '(1)) ; 
()
\end{verbatim}
\end{framed}
Note the important difference: \takeaway{\clocode{first} returns an element, and \clocode{rest} returns a list.} 

Often you need to check if a list (or any other collection) is empty. A collection is \fut{empty} if it does not have any elements. There are several ways to check if a list is empty, but for now we cover just one, the function \clocode{empty?} (recall the convention that names of functions returning a boolean end with a question mark). The function returns \clocode{true} if the list is empty and \clocode{false} otherwise.
\begin{framed}
\begin{verbatim}
intro.core=> (empty? '(5))
false
intro.core=> (empty? '())
true
intro.core=> (empty? (rest '(5)))
true
\end{verbatim}
\end{framed}

\subsection{The very basics of vectors}\label{subsec:vectors}
As we mentioned earlier, vectors are Clojure-specific collection with a faster implementation than lists. They are similar to lists in many ways, so the examples below may look like lists, just with a different notation, although there are a few differences that we will discuss. Section~\ref{subsec:conj-lists-vectors} focuses more on the differences between lists and vectors. 

When creating a vector of particular elements, use square brackets (without a quotation). An empty vector is written as an empty set of square brackets.
\begin{framed}
\begin{verbatim}
intro.core=> [1 2 3] ; creating a vector
[1 2 3]
intro.core=>  ; vectors can have any type of elements: 
intro.core=> [true 6.7 "banana" false]
[true 6.7 "banana" false]
intro.core=> [] ; an empty vector
[]
intro.core=> (def myvector [5 3 4]) ; store vector in a variable 
#'intro.core/myvector
intro.core=> myvector
[5 3 4]
\end{verbatim}
\end{framed}
The same functions  \clocode{first}, \clocode{rest}, and \clocode{empty?} that we have seen for lists work on vectors:

\begin{framed}
\begin{verbatim}
intro.core=> (first myvector)
5
intro.core=> (rest myvector) ; returns a sequence!
(3 4)
intro.core=> (empty? myvector)
false
intro.core=> (empty? [])
true
\end{verbatim}
\end{framed}
The results of   \clocode{first} and  \clocode{empty?}  are as expected. Observe, however, that even though \clocode{rest} returns all but the first elements of the vector (just like for lists), it returns it as something that looks like a list, not a vector: \clocode{(3 4)}. That is because  \clocode{rest}, just like most other functions on collections, returns the result as a sequence, not any specific collection\footnote{In fact, \clocode{rest} for lists also returns a sequence that's not necessarily a list, but sequences are shown just like lists, so there is no way at this point to observe the difference.}. 

This approach allows for a more uniform treatment of collections: since most functions on collections take and return sequences rather than specific collection types (such as lists and vectors), it's much easier to combine them following sequence abstraction rather than being concerned with details of a specific collection type. Sections~\ref{subsec:conj-lists-vectors} and~\ref{subsec:sequences} discuss this approach further. 

To summarize, \takeaway{vectors are similar to lists in how they are created and worked with}, although they have a slightly different syntax.  \takeaway{The function \clocode{rest} on both lists and vectors returns a sequence, not any particular collection type.}


\subsection{\clocode{nil}: representing ``nothing" in Clojure}\label{subsec:nil}
What happens if we call \clocode{first} or \clocode{rest} on an empty list or a vector?  Let us try it in REPL:
\begin{framed}
\begin{verbatim}
intro.core=> (rest '())
()
intro.core=> (rest [])
()
intro.core=> (empty? (rest '()))
true
intro.core=> (empty? (rest []))
true
intro.core=> (first '())
nil
intro.core=> (first [])
nil
\end{verbatim}
\end{framed}
The first four lines of code show that \clocode{rest} of an empty collection (a list or a vector) is an empty sequence \clocode{()}. This makes sense because there are no elements that belong to this sequence. Taking \clocode{first} of an empty sequence is more interesting: it is supposed to return the first element, but there is no element to return, so what should the program do? 

One may think that an attempt to take the first element would result in an error (which is what happens in some other languages), but that's not what happens in Clojure. Instead \clocode{first} of an empty sequence returns a special value \clocode{nil} which is a value that stands for \fut{nothing}. It is often returned from functions that failed to find an elements with desired properties in a sequence. In this case \clocode{first} failed to find the first element, thus \clocode{nil} is a proper thing to return. 

Since we will be encountering \clocode{nil} quite often, it is helpful to study it a bit more. We will also look into the corresponding function \clocode{nil?} that checks whether a given value is nil. Below are some tests that we ran on  \clocode{nil}:
\begin{framed}
\begin{verbatim}
intro.core=> (empty? nil)
true
intro.core=> (nil? '())
false
\end{verbatim}
\end{framed}
We notice that while \clocode{nil} is considered to be empty, an empty list (or any other empty sequence, as you can easily check) is not nil. One may think of  \clocode{nil} as ``emptier than empty''. 

To summarize: \takeaway{\clocode{nil} is a special value in Clojure that means ``nothing at all''. It is often returned from functions that cannot find an element with certain properties in a sequence, such as the first element in an empty sequence. While \clocode{nil} is considered an empty sequence, an empty sequence is not considered nil.}

\subsection{Creating lists and vectors out of existing ones}\label{subsec:conj-lists-vectors}
It is quite common to create collections from existing ones by adding elements to them. Recall that Clojure collections are immutable which means that every time you add an element to a collection you are actually creating a new one, not modifying the existing one. %\elenacomment{add a back reference to discussion of immutability} 
It seems very inefficient, but keep in mind that collections are designed in such a way that, even though a new collection is created with every added (or removed) element, common parts of the old and the new collection are reused (see section~\ref{subsec:implement} for details).  
%\elenacomment{add a forward reference to the corresponding discussion - need some diagrams!}

There are two ways to add an element to a collection: \fut{collection-specific} and \fut{collection-independent}. In a collection-specific approach you create a collection of exactly the same type as the one you are adding to. In a collection-independent approach your result is some type of  sequence, and since all sequences act the same, it doesn't matter how the resulting sequence is implemented internally. Both approaches have their benefits and their drawbacks, as we discuss later. 

\subsubsection{Collection-specific addition of elements to lists and vectors. }\label{subsubsec:conj}
Every Clojure collection supports a function \clocode{conj} that takes this collection and an element and produces a new collection (of the same type) with the element added. In other words, if I add an element using  \clocode{conj} to a vector, I get a new vector, and if I add it to a list, I get a new list. Let's try this in REPL:
\begin{framed}
\begin{verbatim}
intro.core=> (conj '() 5)
(5)
intro.core=> (conj [] 5)
[5]
intro.core=> (conj '(1 5 3) 2)
(2 1 5 3)
intro.core=> (conj [1 5 3] 2)
[1 5 3 2]
\end{verbatim}
\end{framed}
You can see that adding an element \clocode{5} to an empty list creates a list \clocode{(5)}, and adding this element to an empty vector creates a vector \clocode{[5]}, as one would expect. Interestingly, adding an element to a non-empty list or vector using \clocode{conj} results in a different order of elements: adding \clocode{2} to a list \clocode{(1 5 3)} places the new element in the beginning of the list, resulting in \clocode{(2 1 5 3)}, and adding the same element \clocode{2} to a vector with the same elements \clocode{[1 5 3]} results in a vector with the new element added at the end: \clocode{[1 5 3 2]}. Recall that in section~\ref{subsec:coll-seq} we described lists as something similar to a stack of plates, and vectors as similar to a queue in a cafeteria. When you add a new plate to a stack, you put it on top of the stack  (LIFO behavior). When a new person gets into a queue, they get in at the end of the queue (FIFO behavior). The function \clocode{conj} is collection-specific, thus it follows the collection's behavior, producing different results for lists and vectors. 

This difference in behavior has important implications for using \clocode{conj}: if you start with a vector and a list containing the same elements and add the same element to both of them using \clocode{conj}, they will have a different order of elements. For instance: 
\begin{framed}
\begin{verbatim}
intro.core=> (def list1 (conj '(1 5 3) 2))
#'intro.core/list1
intro.core=> (def vec1 (conj [1 5 3] 2))
#'intro.core/vec1
intro.core=> (first list1)
2
intro.core=> (first vec1)
1
\end{verbatim}
\end{framed}
Since \clocode{first} always takes the first, i.e. the leftmost/front/top, element of a collection, it returns different elements for \clocode{list1} and \clocode{vec1}, despite the fact that both were created by adding the same element \clocode{2} to a collection that contained the same initial elements \clocode{1, 5, 3} in the same order. If your program depends on \clocode{first} returning an element at a specific position, switching from lists to vectors (or from vectors to lists) in your program would make a significant difference. 

One may wonder why Clojure has such a difference in behavior of such a fundamental function as \clocode{conj}. The reason for this is that collection-specific functions have the behavior that fits the collections \fut{internal representation} in the best way. In other words, because of the difference in internal representation of lists and vectors it is faster to create a new list with an element added to the front and it is faster to create a new vector with an element added at the end. We give more details about internal representation of lists and vectors in section~\ref{subsec:implement}. Depending on properties of your data, you may choose to use lists or vectors, and \clocode{conj} would guarantee the fastest behavior of your chosen collection\footnote{This is the case not only for lists and vectors, but also for other types of collection that we will see later, such as sets, hash maps, etc. In fact, anyone can define their own collection in Clojure, and provide their own \clocode{conj} that adds an element to it in the way that best fits the collection's internal representation.}. Even if the internal representation of the collection changes, you don't need to change your program in any way to accommodate the change. 

However, the speed of your program is a secondary consideration. The primary one is correctness since an incorrect program is of no use, no matter how fast it is. Using collection-specific functions, such as \clocode{conj}, requires that you constantly think of whether the newly added element appears at the front or at the end, and the confusion between the two may lead to an incorrect program. Since for the most part speed doesn't matter for the programs you are writing at this point, we recommend that you stay away from using \clocode{conj} (and another collection-dependent function, \clocode{into}, described in section~\ref{subsubsec:into-concat}) and use collection-independent functions given in the next section. 

One final important reminder about lists and vectors:  just like all Clojure collections, they are \fut{immutable}, so \clocode{conj} returns a new collection when a new element is added, while the original one remains exactly the same as before:
\begin{framed}
\begin{verbatim}
intro.core=> (def list1 '(1 true "apple"))
#'intro.core/list1
intro.core=> (def list2 (conj list1 5))
#'intro.core/list2
intro.core=> list1
(1 true "apple")
intro.core=> list2
(5 1 true "apple")
\end{verbatim}
\end{framed}
You can see that adding an element \clocode{5} to \clocode{list1} created a new list (which we named \clocode{list2}), but hasn't changed \clocode{list1}. You can check that the same is true for vectors. 

To summarize what we have learned: 
\takeaway{
\begin{itemize}
\item \clocode{conj} is a collection-specific function that adds an element to a collection.
\item \clocode{conj} creates a new collection of the same type (a list for a list and a vector for a vector) that contains all of the elements of the given collection and the new element. 
\item The position of the new element in the resulting collection depends on the type of the given collection: it is added to the beginning of a list in a LIFO manner and to the end of a vector in a FIFO manner. 
\item Collection-specific functions provide the fastest operations on a given collection. 
\item Because the position of the resulting elements is collection-dependent, \clocode{conj} may be confusing to use. 
\end{itemize}
}

%\elenacomment{show a test in REPL that demonstrates that a new list/vector is created and the original is unchanged}

\subsubsection{Collection-independent addition of elements to lists and vectors. }\label{subsubsec:add-first-last}

\elenacomment{add-first, add-last}

\subsubsection{Combining two lists or vectors}\label{subsubsec:into-concat}

\elenacomment{into, concat}

\subsection{Working with sequences}\label{subsec:sequences}

Sequences generally act like lists. 

\elenacomment{introduce \clocode{seq?}}

\subsection{In-depth: implementation of lists and vectors}\label{subsec:implement}

\subsection{Section resources:}
\begin{itemize}
\item \url{http://bpeirce.me/clojure-sequence-implementations.html}
\end{itemize}

