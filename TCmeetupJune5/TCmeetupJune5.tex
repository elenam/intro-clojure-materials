\documentclass{beamer}
\usepackage{beamerthemeshadow}
\usepackage{color}
\usepackage[all]{xy}

\newcommand{\allcomments}[1]{{#1}}
\definecolor{JoesGold}{RGB}{204,102,0}
\definecolor{ForestGreen}{RGB}{34,139,34}
\newcommand{\joecomment}[1]{{\bf \color{JoesGold}{\allcomments{{#1}}}}}
\newcommand{\elenacomment}[1]{{\bf \textcolor{ForestGreen}{\allcomments{{#1}}}}}

\mode<presentation>
{
  \usetheme{Copenhagen} %%% Change later
 \usecolortheme{beaver}


  \setbeamercovered{transparent}
  % or whatever (possibly just delete it)
}
\setbeamertemplate{footline}[page number]{}


\begin{document}
\title{Steps towards using Clojure in an introductory CS class}
\author{Elena Machkasova, Stephen Adams, Joe Einertson}
\institute[UMM] % (optional, but mostly needed)
{
 % \inst{1}%
  University of Minnesota, Morris
}
\date[]  
{ Twin Cities Clojure group meeting, June 5 2013.}

\begin{frame}
  \titlepage
\end{frame}

\begin{frame}

  \frametitle{Outline}
\tableofcontents
\end{frame}


\section{Why Clojure in CS introductory class?}

%\subsection{Current intro class}

\begin{frame}
  \frametitle{Why a functional language in an intro CS class?}
Why teach a functional language in an intro class if mostly imperative languages are used later? 
\begin{itemize}
\item Focus on abstraction, generalization, and modularity. 
\item Focus on functions.
\item Focus on compact, logical code design. 
\item Better understanding of recursion (useful for recursive data structures, e.g. binary trees, and recursive algorithms, e.g. sorting)
\end{itemize}
\end{frame}


\begin{frame}
  \frametitle{Current intro class at UMM}
Currently use How to Design Programs (HtDP) curriculum developed by Matthias Felleisen, Robert Bruce Findler, Matthew Flatt, Shriram Krishnamurthi. 
\begin{itemize}
\item Language: Racket (a version of Scheme).
\item Environment: DrRacket. Features several language levels (Beginner $\to$ Advanced), libraries for incorporating graphics and interaction. 
%\item Online textbook, series of exercises.
\item A large scale open-ended group exercise: develop a game. Interactive, somewhat competitive. Allows students to explore the language and practice/develop design techniques. 
%\item \elenacomment{Something about the timeline?}
\end{itemize}
%%%{\tt add a picture}
\end{frame}

\subsection{Goals and challenges of teaching Clojure in intro class}

\begin{frame}
  \frametitle{Why Clojure in an intro CS class?}
Benefits from switching to Clojure:
\begin{itemize}
\item Rich set of data structures (lists/vectors, hash maps, sets, strings).
\item Rich set of predefined functions (convenience and a good example of language design). 
\item Abstraction (e.g. data structures handled in a uniform manner via sequence abstraction). 
\item Interoperability with Java (may be useful in later courses).
\item Concurrency features (may be useful for individual projects, in future courses, in other contexts). 
\item Used in industry.
\item Used in open source development, has a community around it. 
%\item {\it More?}
\end{itemize}
\end{frame}


\begin{frame}
  \frametitle{Requirements for teaching Clojure in an intro class.}
HtDP (Racket framework) sets the bar high.

What do we need for a good learning experience with Clojure?
\begin{itemize}
\item Environment for new students: convenient text editor (syntax highlighting, autocomplete, code formatting, etc).
\item A project manager: handling dependencies, run/repl, file manager, etc. 
\item Applications: graphics, working with images, etc. 
\item Testing + debugging. 
\item Beginner-friendly error messages. 
\item An approach to introducing Clojure and CS concepts through Clojure. 
\end{itemize}
\end{frame}

\begin{frame}
  \frametitle{Current state of the project.}
What we have or are working on:
\begin{itemize}
\item Text editor: currently jEdit, we hope to use Light Table. 
\item A project manager: leiningen, as a plugin. Command line at this point. 
\item Applications: graphics, working with images, etc.: turtle graphics, seesaw; need beginner-friendly support.
\item Testing + debugging: several options, under development (clojure.test, test.generative, Clojure Debugging Toolkit (CDT), spyscope).
\item{\bf  Beginner-friendly error messages.}
\item {\bf An approach to introducing Clojure and CS concepts through Clojure. }
\end{itemize}
\end{frame}

\section{Beginner-friendly error messages.}

\begin{frame}
  \frametitle{Improving error messages}
Our approach to improving error messages:
\begin{itemize}
\item Run the code inside try/catch, catch exceptions. 
\item Filter the stack, removing {\tt clojure} and {\tt java}. 
\item Wrap common functions (e.g. map) into pre-conditions. 
\item Record information about an argument that failed a pre-condition.
\item Rewrite error messages (pattern-matching). 
\end{itemize}
This is work-in-progress. 
\end{frame}


%\begin{frame}
%  \frametitle{Filtering the stack}
%\begin{itemize}
%\item {\tt clj-stacktrace} library gives access to stack trace info. 
%\item Filter out everything that starts with {\tt clojure} and {\tt java}. 
%\item surround all student code with {\tt try/catch}, catch exceptions, pass through a filtering/prettifying function. 
%\end{itemize}
%%%%{\tt Options, ideally customizable, example}
%\end{frame}

\begin{frame}[fragile]
  \frametitle{Pre-conditions for error messages.}
%\begin{itemize}
\begin{verbatim}
(defn filter [argument1 argument2]
  {:pre [(check-if-function? argument1) 
            (check-if-seqable? argument2)]}
  (clojure.core/filter argument1 argument2))
\end{verbatim}
{\tt check-if-function} uses {\tt fn?} to check if {\tt argument1} is a function. 
If it's not then records what {\tt argument1} is. 

When {\tt AssertionError} is caught, it matches the pattern and extracts the argument info. 
\end{frame}

\begin{frame}[fragile]
  \frametitle{Sample error messages: {\tt filter}}
{\tt (filter 9 [1 2 3])}

\begin{verbatim}
ERROR: First argument 9 must be a function but is a number

Sequence of function calls:
     corefns.core/filter (core.clj line 142)
     intro.core/eval316 (NO_SOURCE_FILE line 313)
     intro.core/test-and-continue (core.clj line 21)
     intro.core/test-filter (core.clj line 313)
     intro.core/-main (core.clj line 372)
     user/eval309 (NO_SOURCE_FILE line 1)
\end{verbatim}
\end{frame}

\begin{frame}[fragile]
  \frametitle{Sample error messages: {\tt filter}}
{\tt (filter odd? *)}

{\tt 
ERROR: Second argument * must be a sequence but is a function
}

{\tt (filter "abc" "123")}

{\tt 
ERROR: First argument "abc" must be a function but is a string}

{\tt (filter (+ 2 3) "123")}

{\tt ERROR: First argument 5 must be a function but is a number}

\end{frame}

\begin{frame}[fragile]
  \frametitle{Handling multiple arguments in asserts.}
\begin{verbatim}
(defn map [argument1 & args]
  {:pre [(check-if-function? argument1) 
            (check-if-seqables? args 2)]}
  (apply clojure.core/map argument1 args))
\end{verbatim}

{\tt (map dec inc)}

{\tt ERROR: Second argument inc must be a sequence but is a function}

{\tt (map + [1 2 3] :a)}

{\tt ERROR: Third argument :a must be a sequence but is a keyword}
\end{frame}

\begin{frame}[fragile]
  \frametitle{Rewriting Clojure error messages.}
Handling messages not generated by asserts:

\vspace{.1in}

{\tt (map + )}

{\it Original:}
{\tt clojure.lang.ArityException Wrong number of args (1) passed to: core\$map}

{\it Rewritten:}
{\tt ERROR: Wrong number of arguments (1)  passed to a function map}

\vspace{.1in}

{\tt (\#(+ \% \%) 1 3))}

{\it Original:}
{\tt clojure.lang.ArityException Wrong number of args (2) passed to: core\$eval325\$fn}

{\it Rewritten:}
{\tt ERROR: Wrong number of arguments (2)  passed to an anonymous function}
 
\end{frame}

\begin{frame}[fragile]
  \frametitle{Rewriting Clojure error messages (cont).}

{\tt (peek \#{1 2 3})}

{\it Original:}
{\tt java.lang.ClassCastException: clojure.lang.PersistentHashSet cannot be cast to clojure.lang.IPersistentStack}

{\it Rewritten:}
{\tt ERROR: Attempted to use a set, but an object that behaves as a stack (such as a vector or a list) was expected.}

\vspace{.1in}

{\tt (assoc "abc" 2 3)}

{\it Original:}
{\tt  java.lang.ClassCastException java.lang.String cannot be cast to clojure.lang.Associative}

{\it Rewritten:}
{\tt ERROR: Attempted to use a string, but a map or a vector was expected.}
\end{frame}

\begin{frame}
  \frametitle{Work in progress}
\begin{itemize}
\item Compilation errors.
\item Laziness (got better with asserts). 
\item Extracting function names (e.g. $\_PLUS\_$).
\item Waaaay more testing. 
\end{itemize}
\end{frame}

\section{How to teach Clojure to beginners.}

\begin{frame}
  \frametitle{Clojure was not made for beginners.}
\begin{itemize}
\item There are many ways to do the same thing.
\item Documentation and examples are confusing.
\item Some functions are confusing, e.g. {\tt contains?, some}.
\item Terminology isn't always clear (collection vs sequence).
\item Need to know behavior of specific collections, e.g. {\tt (into [1 2] \{3 4\})}.
\end{itemize}
\end{frame}

\begin{frame}
  \frametitle{What to do: small(er) stuff.}
\begin{itemize}
\item Introduce material slowly, with examples in lecture notes, etc.
\item Add more intuitive functions: {\tt contains-value?, some?, any?}
\item More predefined functions for strings (not to have to write {\tt apply str}).
\item More wrapper functions for graphics: turtle graphics, seesaw. 
\end{itemize}
\end{frame}

\begin{frame}
  \frametitle{What to do: big picture.}
Collections vs sequences:
\begin{itemize}
\item Collections: have specific purpose and implementations. 
\item Collection-specific functions ({\tt conj, into, seq}) behave differently on different collections:
{\tt (seq [1 2 3 4)} is {\tt (1 2 3 4)}, {\tt (seq \#\{1 2 3 4\}} is {\tt ([1 2] [3 4])}. 
\item {\tt conj, into} return specific collections. 
\item Sequences are an abstract representation. 
\item Collection-independent functions ({\tt map, filter, take, drop} behave the same on any collection (more precisely, on its sequence). 
\item Sequence abstraction is easier to work with. 
\end{itemize}
\end{frame}

\begin{frame}[fragile]
  \frametitle{Motivating example.}
Writing a {\tt reverse} using {\tt reduce}. 
\begin{verbatim}
;; works:
(defn my-reverse [coll]
    (reduce (fn [c x] (conj c x))  () coll)) 

;; doesn't work:
(defn my-reverse [coll]
    (reduce (fn [c x] (conj c x)) [] coll)) 
\end{verbatim}
What is wrong with this?
\end{frame}

\begin{frame}[fragile]
  \frametitle{Motivating example (cont).}
{\tt conj} is collection-specific, i.e. not abstract. Try {\tt cons}:

%Using {\tt cons} instead:
\begin{verbatim}
;; works:
(defn my-reverse [coll]
    (reduce (fn [c x] (cons x c))  () coll)) 

;; works:
(defn my-reverse [coll]
    (reduce (fn [c x] (cons x c)) [] coll)) 
\end{verbatim}
{\tt cons} is collection-independent: works with {\tt (),[],\{\},\#\{\}},...

Returns a sequence, not a specific collection. 

Focus on concepts (what does it mean to reverse a sequence?), not on implementation. 
\end{frame}

\begin{frame}[fragile]
  \frametitle{What to do: big picture.}
\begin{itemize}
\item Make the distinction between collections (concrete) and sequences (abstract) clear. 
\item Encourage use of collection-independent functions (e.g. {\tt concat} over {\tt into}).
\item Add collection-independent functions: {\tt add-first}, {\tt add-last}.
\end{itemize}
Benefit: focusing on concepts, abstraction. 

Added bonus: show the difference between abstractions and concrete implementation?

Yes, but what about efficiency? -- Intro class focuses on concepts. 
\end{frame}

\section{Future work}

\begin{frame}
  \frametitle{Introductory class}

\end{frame}



\begin{frame}
  \frametitle{Discussion}
Questions? Suggestions? Comments? Critique? 
\end{frame}

\end{document}

