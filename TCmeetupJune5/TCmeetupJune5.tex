\documentclass{beamer}
\usepackage{beamerthemeshadow}
\usepackage{color}
\usepackage[all]{xy}

\newcommand{\allcomments}[1]{{#1}}
\definecolor{JoesGold}{RGB}{204,102,0}
\definecolor{ForestGreen}{RGB}{34,139,34}
\newcommand{\joecomment}[1]{{\bf \color{JoesGold}{\allcomments{{#1}}}}}
\newcommand{\elenacomment}[1]{{\bf \textcolor{ForestGreen}{\allcomments{{#1}}}}}

\mode<presentation>
{
  \usetheme{Warsaw} %%% Change later
 \usecolortheme{dove}


  \setbeamercovered{transparent}
  % or whatever (possibly just delete it)
}
\setbeamertemplate{footline}[page number]{}


\begin{document}
\title{Steps towards using Clojure in an introductory CS class}
\author{Elena Machkasova, Stephen Adams, Joe Einertson}
\institute[UMM] % (optional, but mostly needed)
{
 % \inst{1}%
  University of Minnesota, Morris
}
\date[]  
{ Twin Cities Clojure group, June 5 2013.}

\begin{frame}
  \titlepage
\end{frame}

\begin{frame}

  \frametitle{Outline}
\tableofcontents
\end{frame}


\section{Why Clojure in CS introductory class?}

%\subsection{Current intro class}

\begin{frame}
  \frametitle{Why a functional language in an intro CS class?}
Why teach a functional language in an intro class if mostly imperative languages are used later? 
\begin{itemize}
\item Focus on abstraction, generalization, and modularity. 
\item Focus on functions.
\item Focus on compact, logical code design. 
\item Better understanding of recursion (useful for recursive data structures, e.g. binary trees, and recursive algorithms, e.g. sorting)
\end{itemize}
\end{frame}


\begin{frame}
  \frametitle{Current intro class at UMM}
Currently use How to Design Programs (HtDP) curriculum developed by Matthias Felleisen, Robert Bruce Findler, Matthew Flatt, Shriram Krishnamurthi. 
\begin{itemize}
\item Language: Racket (a version of Scheme).
\item Environment: DrRacket. Features several language levels (Beginner $\to$ Advanced), libraries for incorporating graphics and interaction. 
\item Online textbook, series of exercises.
\item A large scale open-ended group exercise: develop a game. Interactive, somewhat competitive. Allows students to explore the language and practice/develop design techniques. 
\item \elenacomment{Something about the timeline?}
\end{itemize}
%%%{\tt add a picture}
\end{frame}

\subsection{Goals and challenges of teaching Clojure in intro class}

\begin{frame}
  \frametitle{Why Clojure in an intro CS class?}
Benefits from switching to Clojure:
\begin{itemize}
\item Rich set of data structures (lists/vectors, hash maps, sets, strings).
\item Rich set of predefined functions (convenience and a good example of language design). 
\item Abstraction (e.g. data structures handled in a uniform manner via sequence abstraction). 
\item Interoperability with Java (may be useful in later courses).
\item Concurrency features (may be useful for individual projects, in future courses, in other contexts). 
\item Used in industry.
\item Used in open source development, has a community around it. 
%\item {\it More?}
\end{itemize}
\end{frame}


\begin{frame}
  \frametitle{Requirements for teaching Clojure in an intro class.}
HtDP (Racket framework) sets the bar high.

What do we need for a good learning experience with Clojure?
\begin{itemize}
\item Environment for new students: convenient text editor (syntax highlighting, autocomplete, code formatting, etc).
\item A project manager: handling dependencies, run/repl, file manager, etc. 
\item Applications: graphics, working with images, etc. 
\item Testing + debugging. 
\item Beginner-friendly error messages. 
\item An approach to introducing Clojure and CS concepts through Clojure. 
\end{itemize}
\end{frame}

\begin{frame}
  \frametitle{Current state of the project.}
What we have or are working on:
\begin{itemize}
\item Text editor: currently jEdit, we hope to use Light Table. 
\item A project manager: leiningen, as a plugin. Command line at this point. 
\item Applications: graphics, working with images, etc.: turtle graphics, seesaw; need beginner-friendly support.
\item Testing + debugging: several options, under development (clojure.test, test.generative, Clojure Debugging Toolkit (CDT), spyscope).
\item{\bf  Beginner-friendly error messages.}
\item {\bf An approach to introducing Clojure and CS concepts through Clojure. }
\end{itemize}
\end{frame}

\section{Beginner-friendly error messages.}

\section{How to teach Clojure to beginners.}


\begin{frame}
  \frametitle{Improving error messages}
Our approach to improving error messages:
\begin{itemize}
\item Filtering the stack. 
\item Wrapping common functions (e.g. map) into pre-conditions with meaningful naming. 
\item Dictionary of common messages.
\end{itemize}
Work in progress, by Joe Einertson (UMM CS'13) and myself. The pre-conditions idea is due to Joe. 
\end{frame}


\begin{frame}
  \frametitle{Filtering the stack}
\begin{itemize}
\item {\tt clj-stacktrace} library gives access to stack trace info. 
\item Filter out everything that starts with {\tt clojure} and {\tt java}. More fine-grained and customizable filtering would be preferred eventually. {\it Any pointers on converting between namespaces and package names?}
\item surround all student code with {\tt try/catch}, catch exceptions, pass through a filtering/prettifying function. 
\end{itemize}
%%%{\tt Options, ideally customizable, example}
\end{frame}

\begin{frame}[fragile]
  \frametitle{Pre-conditions for error messages}
%\begin{itemize}
Wrap functions in {\tt clojure.core} in our function with a pre-condition:
\begin{verbatim}
(ns corefns.core
(:refer-clojure :exclude [map])) ; can't overwrite core 

(def is-function? fn?)
(def is-collection? coll?)

(defn map [argument1 argument2]
{:pre[(is-function? argument1)(is-collection? argument2)]}
  (clojure.core/map argument1 argument2))
\end{verbatim}
%\end{itemize}
%{\tt pre-conditions, try/catch, all the plugins}
\end{frame}

\begin{frame}[fragile]
  \frametitle{Sample error messages}

\end{frame}




\section{Future steps}

\begin{frame}
  \frametitle{Introductory class}
Future work for the introductory class:
\begin{itemize}
\item Finalize the environment, fill in the gaps (error message dictionary, etc.).
\item Examples, exercises, notes. 
\item How much state, how much concurrency?
\item Fall 2013: teach the class.
\item Assessment, modify notes, perhaps a textbook?
\end{itemize}
\end{frame}



\begin{frame}
  \frametitle{Discussion}
Questions? Suggestions? Comments? Critique? 
\end{frame}

\end{document}

