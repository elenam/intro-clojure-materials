\documentclass[submission,copyright,creativecommons]{eptcs}
\providecommand{\event}{TBA} % Name of the event you are submitting to
\usepackage{breakurl}             % Not needed if you use pdflatex only.
\usepackage{color}

\newcommand{\allcomments}[1]{{#1}}
%\newcommand{\allcomments}[1]{}

%% Elena's favorite green (thanks, Fernando!)
\definecolor{ForestGreen}{RGB}{34,139,34}

% Uncomment this if you don't want to show comments
\newcommand{\elenacomment}[1]{{\bf \textcolor{ForestGreen}{\allcomments{{#1}}}}}
\newcommand{\stephencomment}[1]{{\bf \color{blue}{\allcomments{{#1}}}}} %%%  or pick your color
\newcommand{\todo}[1]{{\bf \color{magenta}{\allcomments{ To-do: {#1}}}}}
% \renewcommand{\elenacomment}[1]{}


\title{Benefits of to teaching the Clojure programming language in an introductory CS class and approaches to teaching it.}
\author{Elena Machkasova 
\institute{University of Minnesota, Morris\\
\email{elenam@morris.umn.edu}}
\and
Stephen Adams 
\institute{???}
\email{\quad tba@tba}
}
\def\titlerunning{TBA}
\def\authorrunning{TBA}
\begin{document}
\maketitle

\begin{abstract}
Abstract goes here.
\end{abstract}

\section{Introduction}

\section{Overview and history of Clojure}

\elenacomment{Perhaps combine some sections, make them subsections.}

\section{Benefits of teaching Clojure as the first language}

\todo{Benefits of teaching Clojure to undergrads: provides all the benefits of teaching functional first, integrates with Java,  is used in industry and becoming quite popular, is done right, is a great language to program for oneself, introduces parallel computation}

\section{Technical challenges of teaching Clojure as the first language}

\todo{Error messages, development environment, lack of examples for beginners, some confusing names since they are a part of a full-scale language.}

\section{Approaches to teaching Clojure to beginners}

\todo{Clojure concepts: sequence abstraction vs concrete collections, laziness}


\section{Bibliography}

Sample stuff from example for now.

We request that you use
\href{http://www.cse.unsw.edu.au/~rvg/EPTCS/eptcs.bst}
{\tt $\backslash$bibliographystyle$\{$eptcs$\}$}
\cite{bibliographystylewebpage}. Compared to the original {\LaTeX}
{\tt $\backslash$biblio\-graphystyle$\{$plain$\}$},
it ignores the field {\tt month}, and uses the extra
bibtex fields {\tt eid}, {\tt doi}, {\tt ee} and {\tt url}.
The first is for electronic identifiers (typically the number $n$
indicating the $n^{\rm th}$ paper in an issue) of papers in electronic
journals that do not use page numbers. The other three are to refer,
with life links, to electronic incarnations of the paper.

Almost all publishers use digital object identifiers (DOIs) as a
persistent way to locate electronic publications. Prefixing the DOI of
any paper with {\tt http://dx.doi.org/} yields a URI that resolves to the
current location (URL) of the response page\footnote{Nowadays, papers
  that are published electronically tend
  to have a \emph{response page} that lists the title, authors and
  abstract of the paper, and links to the actual manifestations of
  the paper (e.g.\ as {\tt dvi}- or {\tt pdf}-file). Sometimes
  publishers charge money to access the paper itself, but the response
  page is always freely available.}
of that paper. When the location of the response page changes (for
instance through a merge of publishers), the DOI of the paper remains
the same and (through an update by the publisher) the corresponding
URI will then resolve to the new location. For that reason a reference
ought to contain the DOI of a paper, with a life link to corresponding
URI, rather than a direct reference or link to the current URL of
publisher's response page. This is the r\^ole of the bibtex field {\tt doi}.
DOIs of papers can often be found through
\url{http://www.crossref.org/guestquery};\footnote{For papers that will appear
  in EPTCS and use \href{http://www.cse.unsw.edu.au/~rvg/EPTCS/eptcs.bst}
  {\tt $\backslash$bibliographystyle$\{$eptcs$\}$} there is no need to
  find DOIs on this website, as EPTCS will look them up for you
  automatically upon submission of a first version of your paper;
  these DOIs can then be incorporated in the final version, together
  with the remaining DOIs that need to found at DBLP or publisher's webpages.}
the second method {\it Search on article title}, only using the {\bf
surname} of the first-listed author, works best.  
Other places to find DOIs are DBLP and the response pages for cited
papers (maintained by their publishers).
{\bf EPTCS requires the inclusion of a DOI in each cited paper, when available.}

Often an official publication is only available against payment, but
as a courtesy to readers that do not wish to pay, the authors also
make the paper available free of charge at a repository such as
\url{arXiv.org}. In such a case it is recommended to also refer and
link to the URL of the response page of the paper in such a
repository.  This can be done using the bibtex fields {\tt ee} or {\tt
url}, which are treated as synonyms.  These fields should not be used
to duplicate information that is already provided through the DOI of
the paper.
You can find archival-quality URL's for most recently published papers
in DBLP---they are in the bibtex-field {\tt ee}. In fact, it is often
useful to check your references against DBLP records anyway, or just find
them there in the first place.

When using {\LaTeX} rather than {\tt pdflatex} to typeset your paper, by
default no linebreaking within long URLs is allowed. This leads often
to very ugly output, that moreover is different from the output
generated when using {\tt pdflatex}. This problem is repaired when
invoking \href{http://www.cse.unsw.edu.au/~rvg/EPTCS/breakurl.sty}
{\tt $\backslash$usepackage$\{$breakurl$\}$}: it allows linebreaking
within links and yield the same output as obtained by default with
{\tt pdflatex}. 
When invoking {\tt pdflatex}, the package {\tt breakurl} is ignored.

\nocite{*}
\bibliographystyle{eptcs}
\bibliography{generic}
\end{document}
