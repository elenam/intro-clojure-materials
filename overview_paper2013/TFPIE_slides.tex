\documentclass{beamer}
\usepackage{beamerthemeshadow}
\usepackage{color}
\usepackage[all]{xy}

\newcommand{\allcomments}[1]{{#1}}
\definecolor{JoesGold}{RGB}{204,102,0}
\definecolor{ForestGreen}{RGB}{34,139,34}
\newcommand{\joecomment}[1]{{\bf \color{JoesGold}{\allcomments{{#1}}}}}
\newcommand{\elenacomment}[1]{{\bf \textcolor{ForestGreen}{\allcomments{{#1}}}}}

\mode<presentation>
{
  \usetheme{Copenhagen} %%% Change later
 \usecolortheme{beaver}


  \setbeamercovered{transparent}
  % or whatever (possibly just delete it)
}
\setbeamertemplate{footline}[page number]{}


\begin{document}
\title{Steps towards teaching the Clojure programming language in an introductory CS  class}
\author{Elena Machkasova, Stephen J Adams, Joe Einertson}
\institute[UMM] % (optional, but mostly needed)
{
 % \inst{1}%
  University of Minnesota, Morris
}
\date[May 13, 2013]  
{Trends in Functional Programming in Education (TFPIE) 2013.}

\begin{frame}
  \titlepage
\end{frame}

\begin{frame}

  \frametitle{Outline}
\tableofcontents
\end{frame}

\section{Overview of Clojure}

\begin{frame}
\frametitle{What is Clojure?}
\begin{itemize}
\item Clojure is a LISP.
\item Developed by Rich Hickey, released in 2007, rapidly gaining popularity. 
\item Designed to support concurrency. %\joecomment{Add another word to narrow down - C supports concurrency too, but Clojure has much better support}
\item Provides multiple immutable persistent data structures (lists, vectors, hash maps, sets, etc.).
\item Runs on the JVM, fully integrated with Java. 
\item Provides REPL (Read Eval Print Loop).
\end{itemize}
\end{frame}

\begin{frame}
\frametitle{Why the popularity?}
\begin{itemize}
\item Elegant.
\item Efficient (fast bytecode, efficient implementation of data structures).
\item Convenient and safe efficient multi-threading. 
\item Access to Java (e.g. Clojure seesaw library which uses Java swing). 
\end{itemize}
\end{frame}

\begin{frame}
\frametitle{Why Clojure in intro CS courses?}
\begin{itemize}
\item It's a real-life language done well. 
\item Introduces multiple data structures; abstraction vs implementation. 
\item Can be used in later courses  (concurrency, interoperability with Java, purely functional data structures). 
\item Has a large friendly community (online resources, google groups, open source projects, meetups) - easy to continue on your own. 
\item Can be easily parallelize on multiple cores (no locking, only a tiny change to the program). 
\item Rapidly increasing demand in industry. 
\end{itemize}
\end{frame}


%\section{Clojure at UMM}
\begin{frame}
% Removed periods from frame titles -Joe
\frametitle{Clojure at UMM}
\begin{itemize}
\item UMM (University of Minnesota, Morris) is an undergrad-only liberal arts campus of UMN, has a small, very active CS department. 
\item Included Clojure in upper-division courses (concurrency, functional programming). 
\item Would like to teach it in intro classes (majors, minors, interested individuals).
\item Introductory course focuses on problem solving and key concepts, e.g. abstraction, recursion.
\item Current project: development environment for beginners; developing approaches for teaching Clojure to beginners. 
\end{itemize}
\end{frame}

\section{Technical challenges of teaching Clojure as the first language}

\begin{frame}
\frametitle{Technical challenges of teaching Clojure to beginners}
A need for a beginner-friendly development environment:
\begin{itemize}
\item Text editor.
\item Project manager.
\item Error handling.
\item Some functions behave unexpectedly for beginners. 
\end{itemize}
\end{frame}

\begin{frame}
\frametitle{Development environment: text editor}
Currently there are very few options for a text editor for beginner programmers. 

What doesn't work:
\begin{itemize}
\item Emacs, vim (too complicated for beginners).
\item Eclipse plugin Counterclockwise (too large).
\item Clojure-specific text editors: Clooj, Catnip (too unstable). 
\end{itemize}
What we would like eventually:
\begin{itemize}
\item Light Table: a text editor based on functions, not files; instant evaluation, etc. Still in development. 
\end{itemize}
What we are using:
\begin{itemize}
\item jEdit with LISP/Clojure plugins. 
\end{itemize}
\end{frame}

\begin{frame}
\frametitle{Development environment: project setup}
Clojure projects are managed by a tool called {\tt leiningen}. We need to include beginner-friendly error handling and functions. Program code can be written in a file or typed into REPL. 
%\elenacomment{Better?} \joecomment{Better...I changed the wording a bit more - now I think it reads well}
%\joecomment{Awkward. Reword (semicolon is out of place)}

\vspace*{.1in}

We are developing a {\tt leiningen} plugin for creating and running student projects (work in progress).
%\joecomment{*a* leiningen plugin? Presumably not more than one} \elenacomment{Actually, possibly more than one, but you are right: we don't need to go into details.}
\end{frame}

\begin{frame}
\frametitle{Error handling}
\begin{itemize}
\item Clojure error messages are Java exceptions. 
\item Come with many lines of stack trace. 
\item Refer to Java types. For example, {\tt (cons 2 3)} causes:

{\tt
IllegalArgumentException Don't know how to create ISeq from: java.lang.Long
}
\item We use {\tt try/catch} to catch exceptions and transform them. 
\item We ``filter'' stack trace, leaving only student's code. 
\item We replace types with beginner-friendly ones and rephrase error messages. 
\end{itemize}
\end{frame}

\begin{frame}
\frametitle{Error handling: examples}
Code: {\tt (5 6)} 

Original:

{\tt
java.lang.ClassCastException: java.lang.Long cannot be cast to clojure.lang.IFn
}

Transformed:

{\tt
Error: Attempted to use a number, but a function was expected.
}

Code: {\tt ([1 3 2] 5)} (trying to access an element at index 5 in a 3-element vector).

Original:

{\tt
java.lang.IndexOutOfBoundsException
}

Transformed:

{\tt
Error: An index in a sequence is out of bounds
}
\end{frame}

\begin{frame}
\frametitle{Error handling: work in progress}
Current work in progress:
\begin{itemize}
\item Provide error handling for code typed in REPL.
\item Handle compilation errors. 
\item Developing leiningen plugin to run all student code (file and REPL) inside try/catch. 
\item Provide hints and examples for error messages (``perhaps you swapped the order of arguments?'')
\end{itemize}
\end{frame}

\section{Approaches to teaching Clojure to beginners}

\begin{frame}
\frametitle{Collections vs sequence abstraction}
Clojure collections (lists, vectors, hash maps, sets, etc.): 
\begin{itemize}
\item ...are stored in a way that optimizes their intended use. 
\begin{itemize}
\item lists have constant access time to the beginning and linear to the end. 
\item vectors are shallow trees, provide logarithmic access to any position.
\end{itemize}
\item ...have a few functions specialized to a collection type, e.g. {\tt conj} that returns a collection of the same type with a new element added. 
\begin{itemize}
\item lists: {\tt (conj '(2 3 1) 4)} results in a list {\tt (4 2 3 1)}.
\item vectors: {\tt (conj [2 3 1] 4)} results in a vector {\tt [2 3 1 4]}.
\end{itemize}
\end{itemize}
The difference in behavior is likely to be confusing to beginners. 
\end{frame}

\begin{frame}
\frametitle{Collections vs sequence abstraction (cont.)}
Sequences are an abstraction for a number of elements (possibly infinite) in a specific order. 
\begin{itemize}
\item Most Clojure functions work on sequences and return sequences (e.g. {\tt map}). 
\item It is easier for beginners to program in a collection-independent (i.e. abstract) way.
\item We provide several functions that work in a collection-independent way. They return sequences (look like lists):
\begin{itemize}
\item {\tt (add-first '(2 3 1) 4)} results in a sequence {\tt (4 2 3 1)}.
\item {\tt (add-first [2 3 1] 4)} results in a sequence {\tt (4 2 3 1)}.
\item {\tt (add-last '(2 3 1) 4)} results in a sequence {\tt (2 3 1 4)}.
\item {\tt (add-last [2 3 1] 4)} results in a sequence {\tt (2 3 1 4)}.
\end{itemize}
\end{itemize}
\end{frame}

\begin{frame}[fragile]
\frametitle{Collections vs sequence abstraction (example)}
Define a function to reverse a sequence, using {\tt reduce}  ({\tt fold}). 

Note: {\tt defn} = define function, {\tt fn} = anonymous function (lambda),  {\tt '()} = empty list, {\tt []} = empty vector.  %\joecomment{You may want to include some of these earlier - the list/vector syntax in particular will be confusing on the previous slide} \elenacomment{I will go over the list/vector syntax on the previous slide, this is just a reminder.}
%%{\tiny
%\joecomment{Think you're missing the ' before the () on the last example?}
\begin{verbatim}
;; works because conj adds to the beginning of a list
(defn my-reverse [coll]
    (reduce (fn [c x] (conj c x))  '() coll)) 

;; doesn't work because conj adds at the end of a vector
(defn my-reverse [coll]
    (reduce (fn [c x] (conj c x)) [] coll)) 

;; abstract approach (works with a list or a vector)
(defn my-reverse [coll]
    (reduce (fn [c x] (add-first x c)) '() coll)) 
\end{verbatim} %\joecomment{Think you're missing the ' before the () on the last example?}
%%} %%% end tiny
\end{frame}

\begin{frame}
\frametitle{Abstraction-based teaching approach}
\begin{itemize}
\item Students will see both collection-specific and collection-independent functions. 
\item Collection-independent functions allow focus on problem-solving, make things easier. 
\item Understanding the differences between implementation details (collections) and abstraction (sequence) can carry on to Data Structures and Software Development. %\joecomment{Capitalize DS and SoftwareD so it's obvious they're classes in this context?}
\end{itemize}
\end{frame}

\section{Conclusions}

\begin{frame}
\frametitle{Benefits and challenges of teaching Clojure in intro classes}
Benefits:
\begin{itemize}
\item Clojure has a rich collection of data structures.
\item Based on abstraction, teaches good programming skills. 
\item Used in industry and has a well-developed friendly community. 
\item Provides opportunities for parallelization. 
\end{itemize}
Challenges:
\begin{itemize}
\item Development of beginner-friendly development environment. 
\item Handling error messages.
\item Developing approaches to teaching that present the strengths of Clojure without confusing beginners. 
\item Developing beginner-friendly documentation and examples. 
%Online Clojure documentation and examples is not designed for beginners. 
\end{itemize}% \joecomment{Is it worth mentioning the crappy Clojure documentation as a downside?}
\end{frame}

\begin{frame}
\frametitle{Acknowledgments and selected references}
Selected references:
\begin{itemize}
\item Richard Brown, Elizabeth Shoop, et al: Strategies for preparing computer science students for the multicore world.  ITiCSE working group reports, 2010.
\item Matthias Felleisen, Robert Bruce Findler, Matthew Flatt, Shriram Krishnamurthi: How to design
programs. MIT Press, 2001.
\item Rich Hickey: The Clojure programming language. Symposium on
Dynamic languages, 2008. 
\item Simon Thompson, Steve Hill: Functional programming through the curriculum. Functional
Programming Languages in Education, 1995. 
\end{itemize}
The authors thank Jon Anthony, Brian Goslinga, Nic McPhee, and Simon Hawkin for helpful discussion and Max Magnuson and Paul Schliep for help in testing. 
\end{frame}


\end{document}

