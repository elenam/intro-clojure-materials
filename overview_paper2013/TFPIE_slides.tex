\documentclass{beamer}
\usepackage{beamerthemeshadow}
\usepackage{color}
\usepackage[all]{xy}



\mode<presentation>
{
  \usetheme{Warsaw} %%% Change later
 \usecolortheme{dove}


  \setbeamercovered{transparent}
  % or whatever (possibly just delete it)
}
\setbeamertemplate{footline}[page number]{}


\begin{document}
\title{Steps towards teaching the Clojure programming language in an introductory CS  class}
\author{Elena Machkasova, Stephen J Adams, Joe Einertson}
\institute[UMM] % (optional, but mostly needed)
{
 % \inst{1}%
  University of Minnesota, Morris
}
\date[May 13, 2013]  
{Trends in Functional Programming in Education (TFPIE) 2013.}

\begin{frame}
  \titlepage
\end{frame}

\begin{frame}

  \frametitle{Outline}
\tableofcontents
\end{frame}

\section{Overview of Clojure}

\begin{frame}
\frametitle{What is Clojure?}
\begin{itemize}
\item Clojure is a LISP.
\item Developed by Rich Hickey, released in 2007, rapidly gaining popularity. 
\item Supports concurrency.
\item Provides multiple immutable persistent data structures (lists, vectors, hash maps, sets, etc.).
\item Runs on the JVM, fully integrated with Java. 
\item Provides REPL (Read Eval Print Loop).
\end{itemize}
\end{frame}

\begin{frame}
\frametitle{Why the popularity?}
\begin{itemize}
\item Elegant.
\item Efficient (fast bytecode, tail recursion optimization).
\item Convenient and safe efficient multi-threading. 
\item Access to Java (e.g. Clojure seesaw library which uses Java swing). 
\end{itemize}
\end{frame}

\begin{frame}
\frametitle{Why Clojure in intro CS courses?}
\begin{itemize}
\item It's a real-life language done well. 
\item Introduces multiple data structures; abstraction vs implementation. 
\item Can be used in later courses  (concurrency, interoperability with Java, purely functional data structures). 
\item Has a friendly community (online resources, google groups, open source projects, meetups) - easy to continue on your own. 
\item Rapidly increasing demand in industry. 
\end{itemize}
\end{frame}

%\section{Clojure at UMM}
\begin{frame}
\frametitle{Clojure at UMM.}
\begin{itemize}
\item UMM (University of Minnesota, Morris) is an undergrad-only liberal arts campus of UMN, has a small, very active CS department. 
\item Included Clojure in upper-division courses (concurrency, functional programming). 
\item Would like to teach it in intro classes (majors, minors, interested individuals).
\item Introductory course focuses on problem solving and key concepts, e.g. abstraction, recursion.
\item Current project: development environment for beginners; developing approaches for teaching Clojure to beginners. 
\end{itemize}
\end{frame}

\section{Technical challenges of teaching Clojure as the first language}

\begin{frame}
\frametitle{Development environment: text editor.}
Currently there are very few options for a text editor for beginner programmers. 

What doesn't work:
\begin{itemize}
\item Emacs, vim (too complicated for beginners).
\item Eclipse plugin Counterclockwise (too large).
\item Clojure-specific text editors: Clooj, Catnip (too unstable). 
\end{itemize}
What we would like eventually:
\begin{itemize}
\item Light Table: a text editor based on functions, not files; instant evaluation, etc. Still in development. 
\end{itemize}
What we are using:
\begin{itemize}
\item jEdit with LISP/Clojure plugins. 
\end{itemize}
\end{frame}

\begin{frame}
\frametitle{Development environment: project setup.}
Clojure projects are managed by a tool {\tt leiningen}. We need to include beginner-friendly error handling and beginner-friendly functions; intercept code typed into REPL. 

We are developing {\tt leiningen} plugins for creating and running student projects (work in progress). 
\end{frame}

\begin{frame}
\frametitle{Error handling.}

\end{frame}

\section{Approaches to teaching Clojure to beginners}

\begin{frame}
\frametitle{Collections vs sequence abstraction}
Can I add a reverse example???
\end{frame}

\begin{frame}
\frametitle{Example, maybe}
Can I add a reverse example???
\end{frame}

\end{document}

